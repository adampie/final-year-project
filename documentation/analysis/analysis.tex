\documentclass[12pt,a4paper]{report}
\usepackage[utf8]{inputenc}
\usepackage{amsmath}
\usepackage{amsfonts}
\usepackage{amssymb}
\usepackage{hyperref}
\usepackage{graphicx}
\usepackage[numbers]{natbib}
\usepackage{hyperref}
\usepackage{float}
\usepackage{listings}
\usepackage{color}
\def\UrlBreaks{\do\/\do-}
\graphicspath{ {images/} }
\lstset{language=bash}
\usepackage[a4paper,top=3cm,bottom=2cm,left=3cm,right=3cm,marginparwidth=1.75cm]{geometry}

\begin{document}
\sloppy
\begin{titlepage}
	\centering
	\includegraphics[width=0.15\textwidth]{Brighton-University-logo.png}\par
	{\scshape\LARGE University of Brighton\par}
	\vspace{1cm}
	{\scshape\Large \par}
	\vspace{1.5cm}
	{\huge\bfseries Project Analysis\par}
	\vspace{2cm}
	{\Large\itshape Adam Pietrzycki\par}
	\vfill
	\par
	\textsc{}
	\vfill
	{\large \today\par}
\end{titlepage}

\begin{abstract}
In this document I will be keeping a log of my approach and analysis of this project, note that not everything in this will be of any use.     
\end{abstract}

\pagebreak
\tableofcontents
\pagebreak

\chapter{Original pi-gen}
\section{/}
The build.sh file is the one you run to start generating the Raspbian images. It first sets up a few EXPORTS and SOURCES files from the scripts folder.
\subsection{config}
In the config file you can set an IMG\_NAME and an APT\_PROXY. The default file can just contain "IMG\_NAME='Raspbian'". A quick thing you can do to set up this file is: 
\begin{lstlisting}[frame=single]
echo "IMG_NAME='Raspbian'" > config

NOTE:
>  is overwrite if present, create if not.
>> is add to end of file it present, create if not.
\end{lstlisting}

\subsection{build.sh}
The stages are initially run by this:
\begin{lstlisting}[frame=single, language=bash]
for STAGE_DIR in ${BASE_DIR}/stage*; do
	run_stage
done
\end{lstlisting}
For-each loop, runs run\_stage method in each directory. This helps us as we can change the main script that is run in each folder to run our integrated Ansible code.
\subsubsection{Exports and Source}
A bunch of EXPORTS, most important ones are the directories:
\begin{lstlisting}[frame=single]
export BASE_DIR="$(cd "$(dirname "${BASH_SOURCE[0]}")"&&pwd)"
export SCRIPT_DIR="${BASE_DIR}/scripts"
export WORK_DIR="${BASE_DIR}/work/${IMG_DATE}-${IMG_NAME}"
export DEPLOY_DIR="${BASE_DIR}/deploy"
export LOG_FILE="${WORK_DIR}/build.log"
\end{lstlisting}
This will let us set up static directories when testing the Ansible code, after that can be changed to use dynamic folders.
\subsubsection{Stage}
This seems to \textit{pushd} the previous stage location to /dev/null which clears it? Then assigns new directories. If CLEAN is set to 1 then it will delete the rootfs folder, I presume this is done during the last stage? If there is a prerun.sh in the current working directory then that will run first. Then it will run run\_sub\_stage() in each subdirectory of each stage. Finally it unmounts the stage and assigns a few variables; then \textit{popd} into /dev/null?
\begin{itemize}
\item{\url{http://ss64.com/bash/pushd.html}}
\item{\url{http://ss64.com/bash/popd.html}}
\end{itemize}
\subsubsection{Sub-Stage}
Starts off with \textit{pushd} sub\_stage\_dir again, then a for loop between 00 and 99. First it pre-seeds the \textit{debconf} and after installs packages-nr using \textit{apt-get}, then packages once packages-nr is done. If there are any patches at this point then this is when they are applied. A few things are done with quilt (to do with patches it seems)??? Coming towards the end of sub-stage now any run.sh and run-chroot.sh files are ran. \textit{Popd} is used again.
\begin{itemize}
\item{\url{http://man.he.net/man1/debconf-set-selections}}
\item{\url{https://linux.die.net/man/8/apt-get}}
\item{\url{https://linux.die.net/man/1/quilt}}
\end{itemize}
\subsubsection{Export-images}

\section{scripts/}
\subsection{common}
\subsubsection{log}
Gets current time and uses a pipe with \textit{tee} to write to the log file. 
\\(\url{http://man7.org/linux/man-pages/man1/tee.1.html)}
\subsubsection{bootstrap}
Sets up debootstrap, uses \textit{capsh} to create env I think? 
\\(\url{http://man7.org/linux/man-pages/man1/capsh.1.html})
\subsubsection{copy\_previous}
If rootfs folder doesn't exist it will create one, if it does then it uses \textit{rsync} to copy from previous to current stage. This can be avoided to speed things up?
\\(\url{http://linuxcommand.org/man_pages/rsync1.html})
\subsubsection{unmount}
Does a few checks using \$1, unmouts mounted folders using \textit{umount}.
\\(\url{http://man7.org/linux/man-pages/man8/umount.8.html})
\subsubsection{unmount\_image}
First syncs then get \textit{losetup}, then it does a loop through the directories and uses \textit{unmount()}, finally \textit{kpartx} and then \textit{losetup} again.
\begin{itemize}
\item{\url{http://linuxcommand.org/man_pages/losetup8.html}}
\item{\url{http://www.dsm.fordham.edu/cgi-bin/man-cgi.pl?topic=kpartx&ampsect=8}}
\end{itemize}
\subsubsection{onchroot}
Mounts with bind, uses \textit{realpath and \textit{capsh} again}:
\begin{itemize}
\item{\${ROOTFS\_DIR}/proc}
\item{\${ROOTFS\_DIR}/dev}
\item{\${ROOTFS\_DIR}/dev/pts}
\item{\${ROOTFS\_DIR}/sys}
\end{itemize}
(\url{http://man7.org/linux/man-pages/man3/realpath.3.html})
\subsubsection{update\_issue}
Prints pi-gen version? This is strange; look into it but it does not look like a priority. 
\subsection{dependencies\_check}
Dependencies\_check, checks if each required tool is installed on the system, the list of packages required can be found in the root directory in  ../DEPENDS file.

\section{export-image/}
\section{export-noobs/}
\section{stage0/}
\section{stage1/}
\section{stage2/}
\section{stage3/}
\section{stage4/}

\chapter{Ansible pi-gen}

\chapter{NB}
\section{Kernel panic when Virtualizing}
Different versions of pi-gen would fail at different times, normally in stage 4. This was due to code being reverted thus removing the setting a max build stage functionality. The bento boxes are the safest to use though need to install a few more packages, they come with a 50GB virtual disk where as other Vagrant images came with the standard of 8GB and it was a pain to increase. 
\section{NR}
NR is apt-get --




\end{document}